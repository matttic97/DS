%%%%%%%%%%%%%%%%%%%%%%%%%%%%%%%%%%%%%%%%
% datoteka diploma-vzorec.tex
%
% vzorčna datoteka za pisanje diplomskega dela v formatu LaTeX
% na UL Fakulteti za računalništvo in informatiko
%
% vkup spravil Gašper Fijavž, december 2010
% 
%
%
% verzija 12. februar 2014 (besedilo teme, seznam kratic, popravki Gašper Fijavž)
% verzija 10. marec 2014 (redakcijski popravki Zoran Bosnić)
% verzija 11. marec 2014 (redakcijski popravki Gašper Fijavž)
% verzija 15. april 2014 (pdf/a 1b compliance, not really - just claiming, Damjan Cvetan, Gašper Fijavž)
% verzija 23. april 2014 (privzeto cc licenca)
% verzija 16. september 2014 (odmiki strain od roba)
% verzija 28. oktober 2014 (odstranil vpisno številko)
% verija 5. februar 2015 (Literatura v kazalu, online literatura)
% verzija 25. september 2015 (angl. naslov v izjavi o avtorstvu)
% verzija 26. februar 2016 (UL izjava o avtorstvu)
% verzija 16. april 2016 (odstranjena izjava o avtorstvu)
% verzija 5. junij 2016 (Franc Solina dodal vrstice, ki jih je označil s svojim imenom)


\documentclass[a4paper, 12pt]{book}
%\documentclass[a4paper, 12pt, draft]{book}  Nalogo preverite tudi z opcijo draft, ki vam bo pokazala, katere vrstice so predolge!



\usepackage[utf8x]{inputenc}   % omogoča uporabo slovenskih črk kodiranih v formatu UTF-8
\usepackage[slovene,english]{babel}    % naloži, med drugim, slovenske delilne vzorce
\usepackage[pdftex]{graphicx}  % omogoča vlaganje slik različnih formatov
\usepackage{fancyhdr}          % poskrbi, na primer, za glave strani
\usepackage{amssymb}           % dodatni simboli
\usepackage{amsmath}           % eqref, npr.
%\usepackage{hyperxmp}
\usepackage[hyphens]{url}  % dodal Solina
\usepackage{comment}       % dodal Solina

\usepackage[pdftex, colorlinks=true,
						citecolor=black, filecolor=black, 
						linkcolor=black, urlcolor=black,
						pagebackref=false, 
						pdfproducer={LaTeX}, pdfcreator={LaTeX}, hidelinks]{hyperref}

\usepackage{color}       % dodal Solina
\usepackage{soul}       % dodal Solina

%%%%%%%%%%%%%%%%%%%%%%%%%%%%%%%%%%%%%%%%
%	DIPLOMA INFO
%%%%%%%%%%%%%%%%%%%%%%%%%%%%%%%%%%%%%%%%
\newcommand{\ttitle}{Optična razpoznava notnih znakov}
\newcommand{\ttitleEn}{Optical music notations recognition}
\newcommand{\tsubject}{\ttitle}
\newcommand{\tsubjectEn}{\ttitleEn}
\newcommand{\tauthor}{Matic Isovski}
\newcommand{\tkeywords}{razpoznava, tehnologija, Python}
\newcommand{\tkeywordsEn}{recognition, technology, Python}


%%%%%%%%%%%%%%%%%%%%%%%%%%%%%%%%%%%%%%%%
%	HYPERREF SETUP
%%%%%%%%%%%%%%%%%%%%%%%%%%%%%%%%%%%%%%%%
\hypersetup{pdftitle={\ttitle}}
\hypersetup{pdfsubject=\ttitleEn}
\hypersetup{pdfauthor={\tauthor, matjaz.kralj@fri.uni-lj.si}}
\hypersetup{pdfkeywords=\tkeywordsEn}


 


%%%%%%%%%%%%%%%%%%%%%%%%%%%%%%%%%%%%%%%%
% postavitev strani
%%%%%%%%%%%%%%%%%%%%%%%%%%%%%%%%%%%%%%%%  

\addtolength{\marginparwidth}{-20pt} % robovi za tisk
\addtolength{\oddsidemargin}{40pt}
\addtolength{\evensidemargin}{-40pt}

\renewcommand{\baselinestretch}{1.3} % ustrezen razmik med vrsticami
\setlength{\headheight}{15pt}        % potreben prostor na vrhu
\renewcommand{\chaptermark}[1]%
{\markboth{\MakeUppercase{\thechapter.\ #1}}{}} \renewcommand{\sectionmark}[1]%
{\markright{\MakeUppercase{\thesection.\ #1}}} \renewcommand{\headrulewidth}{0.5pt} \renewcommand{\footrulewidth}{0pt}
\fancyhf{}
\fancyhead[LE,RO]{\sl \thepage} 
%\fancyhead[LO]{\sl \rightmark} \fancyhead[RE]{\sl \leftmark}
\fancyhead[RE]{\sc \tauthor}              % dodal Solina
\fancyhead[LO]{\sc Diplomska naloga}     % dodal Solina


\newcommand{\BibTeX}{{\sc Bib}\TeX}

%%%%%%%%%%%%%%%%%%%%%%%%%%%%%%%%%%%%%%%%
% naslovi
%%%%%%%%%%%%%%%%%%%%%%%%%%%%%%%%%%%%%%%%  


\newcommand{\autfont}{\Large}
\newcommand{\titfont}{\LARGE\bf}
\newcommand{\clearemptydoublepage}{\newpage{\pagestyle{empty}\cleardoublepage}}
\setcounter{tocdepth}{1}	      % globina kazala

%%%%%%%%%%%%%%%%%%%%%%%%%%%%%%%%%%%%%%%%
% konstrukti
%%%%%%%%%%%%%%%%%%%%%%%%%%%%%%%%%%%%%%%%  
\newtheorem{izrek}{Izrek}[chapter]
\newtheorem{trditev}{Trditev}[izrek]
\newenvironment{dokaz}{\emph{Dokaz.}\ }{\hspace{\fill}{$\Box$}}

%%%%%%%%%%%%%%%%%%%%%%%%%%%%%%%%%%%%%%%%%%%%%%%%%%%%%%%%%%%%%%%%%%%%%%%%%%%%%%%
%% PDF-A
%%%%%%%%%%%%%%%%%%%%%%%%%%%%%%%%%%%%%%%%%%%%%%%%%%%%%%%%%%%%%%%%%%%%%%%%%%%%%%%


%%%%%%%%%%%%%%%%%%%%%%%%%%%%%%%%%%%%%%%% 
% define medatata
%%%%%%%%%%%%%%%%%%%%%%%%%%%%%%%%%%%%%%%% 
\def\Title{\ttitle}
\def\Author{\tauthor, mi6568@student.uni-lj.si}
\def\Subject{\ttitleEn}
\def\Keywords{\tkeywordsEn}

%%%%%%%%%%%%%%%%%%%%%%%%%%%%%%%%%%%%%%%% 
% \convertDate converts D:20080419103507+02'00' to 2008-04-19T10:35:07+02:00
%%%%%%%%%%%%%%%%%%%%%%%%%%%%%%%%%%%%%%%% 
\def\convertDate{%
    \getYear
}

{\catcode`\D=12
 \gdef\getYear D:#1#2#3#4{\edef\xYear{#1#2#3#4}\getMonth}
}
\def\getMonth#1#2{\edef\xMonth{#1#2}\getDay}
\def\getDay#1#2{\edef\xDay{#1#2}\getHour}
\def\getHour#1#2{\edef\xHour{#1#2}\getMin}
\def\getMin#1#2{\edef\xMin{#1#2}\getSec}
\def\getSec#1#2{\edef\xSec{#1#2}\getTZh}
\def\getTZh +#1#2{\edef\xTZh{#1#2}\getTZm}
\def\getTZm '#1#2'{%
    \edef\xTZm{#1#2}%
    \edef\convDate{\xYear-\xMonth-\xDay T\xHour:\xMin:\xSec+\xTZh:\xTZm}%
}

\expandafter\convertDate\pdfcreationdate 

%%%%%%%%%%%%%%%%%%%%%%%%%%%%%%%%%%%%%%%%
% get pdftex version string
%%%%%%%%%%%%%%%%%%%%%%%%%%%%%%%%%%%%%%%% 
\newcount\countA
\countA=\pdftexversion
\advance \countA by -100
\def\pdftexVersionStr{pdfTeX-1.\the\countA.\pdftexrevision}


%%%%%%%%%%%%%%%%%%%%%%%%%%%%%%%%%%%%%%%%
% XMP data
%%%%%%%%%%%%%%%%%%%%%%%%%%%%%%%%%%%%%%%%  
\usepackage{xmpincl}
\includexmp{pdfa-1b}

%%%%%%%%%%%%%%%%%%%%%%%%%%%%%%%%%%%%%%%%
% pdfInfo
%%%%%%%%%%%%%%%%%%%%%%%%%%%%%%%%%%%%%%%%  
\pdfinfo{%
    /Title    (\ttitle)
    /Author   (\tauthor, damjan@cvetan.si)
    /Subject  (\ttitleEn)
    /Keywords (\tkeywordsEn)
    /ModDate  (\pdfcreationdate)
    /Trapped  /False
}


%%%%%%%%%%%%%%%%%%%%%%%%%%%%%%%%%%%%%%%%%%%%%%%%%%%%%%%%%%%%%%%%%%%%%%%%%%%%%%%
%%%%%%%%%%%%%%%%%%%%%%%%%%%%%%%%%%%%%%%%%%%%%%%%%%%%%%%%%%%%%%%%%%%%%%%%%%%%%%%

\begin{document}
\selectlanguage{slovene}
\frontmatter
\setcounter{page}{1} %
\renewcommand{\thepage}{}       % preprecimo težave s številkami strani v kazalu
\newcommand{\sn}[1]{"`#1"'}                    % dodal Solina (slovenski narekovaji)

%%%%%%%%%%%%%%%%%%%%%%%%%%%%%%%%%%%%%%%%
%naslovnica
 \thispagestyle{empty}%
   \begin{center}
    {\large\sc Univerza v Ljubljani\\%
      Fakulteta za računalništvo in informatiko}%
    \vskip 10em%
    {\autfont \tauthor\par}%
    {\titfont \ttitle \par}%
    {\vskip 3em \textsc{DIPLOMSKO DELO\\[5mm]         % dodal Solina za ostale študijske programe
%    VISOKOŠOLSKI STROKOVNI ŠTUDIJSKI PROGRAM\\ PRVE STOPNJE\\ RAČUNALNIŠTVO IN INFORMATIKA}\par}%
    UNIVERZITETNI  ŠTUDIJSKI PROGRAM\\ PRVE STOPNJE\\ RAČUNALNIŠTVO IN INFORMATIKA}\par}%
%    INTERDISCIPLINARNI UNIVERZITETNI\\ ŠTUDIJSKI PROGRAM PRVE STOPNJE\\ RAČUNALNIŠTVO IN MATEMATIKA}\par}%
%    INTERDISCIPLINARNI UNIVERZITETNI\\ ŠTUDIJSKI PROGRAM PRVE STOPNJE\\ UPRAVNA INFORMATIKA}\par}%
%    INTERDISCIPLINARNI UNIVERZITETNI\\ ŠTUDIJSKI PROGRAM PRVE STOPNJE\\ MULTIMEDIJA}\par}%
    \vfill\null%
    {\large \textsc{Mentor}: doc.\ dr.  Luka Šajn\par}%
    {\vskip 2em \large Ljubljana, 2020 \par}%
\end{center}
% prazna stran
%\clearemptydoublepage      % dodal Solina (izjava o licencah itd. se izpiše na hrbtni strani naslovnice)

%%%%%%%%%%%%%%%%%%%%%%%%%%%%%%%%%%%%%%%%
%copyright stran
\thispagestyle{empty}
\vspace*{8cm}

\noindent
{\sc Copyright}. 
Rezultati diplomske naloge so intelektualna lastnina avtorja in Fakultete za računalništvo in informatiko Univerze v Ljubljani.
Za objavo in koriščenje rezultatov diplomske naloge je potrebno pisno privoljenje avtorja, Fakultete za računalništvo in informatiko ter mentorja.

\begin{center}
\mbox{}\vfill
\emph{Besedilo je oblikovano z urejevalnikom besedil \LaTeX.}
\end{center}
% prazna stran
\clearemptydoublepage

%%%%%%%%%%%%%%%%%%%%%%%%%%%%%%%%%%%%%%%%
% stran 3 med uvodnimi listi
\thispagestyle{empty}
\vspace*{4cm}

\noindent
Fakulteta za računalništvo in informatiko izdaja naslednjo nalogo:
\medskip
\begin{tabbing}
\hspace{32mm}\= \hspace{6cm} \= \kill




Tematika naloge:
\end{tabbing}
Besedilo teme diplomskega dela študent prepiše iz študijskega informacijskega sistema, kamor ga je vnesel mentor. V nekaj stavkih bo opisal, kaj pričakuje od kandidatovega diplomskega dela. Kaj so cilji, kakšne metode uporabiti, morda bo zapisal tudi ključno literaturo.
\vspace{15mm}






\vspace{2cm}

% prazna stran
\clearemptydoublepage

% zahvala
%\thispagestyle{empty}\mbox{}\vfill\null\it%
%\noindent
%\rm\normalfont

% prazna stran
%\clearemptydoublepage

%%%%%%%%%%%%%%%%%%%%%%%%%%%%%%%%%%%%%%%%
% posvetilo, če sama zahvala ne zadošča :-)
%\thispagestyle{empty}\mbox{}{\vskip0.20\textheight}\mbox{}\hfill%%\begin{minipage}{0.55\textwidth}%
%Svoji dragi Alenčici.
%\normalfont\end{minipage}

% prazna stran
%\clearemptydoublepage


%%%%%%%%%%%%%%%%%%%%%%%%%%%%%%%%%%%%%%%%
% kazalo
\pagestyle{empty}
\def\thepage{}% preprecimo tezave s stevilkami strani v kazalu
\tableofcontents{}


% prazna stran
\clearemptydoublepage

%%%%%%%%%%%%%%%%%%%%%%%%%%%%%%%%%%%%%%%%
% seznam kratic

\chapter*{Seznam uporabljenih kratic}  % spremenil Solina, da predolge vrstice ne gredo preko desnega roba

\begin{comment}
\begin{tabular}{l|l|l}
  {\bf kratica} & {\bf angleško} & {\bf slovensko} \\ \hline
  % after \\: \hline or \cline{col1-col2} \cline{col3-col4} ...
  {\bf CA} & classification accuracy & klasifikacijska točnost \\
  {\bf DBMS} & database management system & sistem za upravljanje podatkovnih baz \\
  {\bf SVM} & support vector machine & metoda podpornih vektorjev \\
  \dots & \dots & \dots \\
\end{tabular}
\end{comment}

\noindent\begin{tabular}{p{0.1\textwidth}|p{.4\textwidth}|p{.4\textwidth}}    % po potrebi razširi prvo kolono tabele na račun drugih dveh!
  {\bf kratica} & {\bf angleško}                             & {\bf slovensko} \\ \hline
  {\bf OCR}      & Optical character recognition              & Optična razpoznava znakov \\
  {\bf OMR}      & Optical music recognition               & Optična razpoznava glasbenih notacij \\
%  \dots & \dots & \dots \\
\end{tabular}


% prazna stran
\clearemptydoublepage

%%%%%%%%%%%%%%%%%%%%%%%%%%%%%%%%%%%%%%%%
% povzetek
\addcontentsline{toc}{chapter}{Povzetek}
\chapter*{Povzetek}

\noindent\textbf{Naslov:} \ttitle
\bigskip

\noindent\textbf{Avtor:} \tauthor
\bigskip

%\noindent\textbf{Povzetek:} 
\noindent V diplomski nalogi bo predstavljen problem razpoznave glasbenih notacij. Vsebovala bo opis in razlago OMR (optical music recognition) tehnologije ob primerih. Rezultat naloge bo mobilna aplikacija, ki bo omogočala slikanje not, ter izbrati znano/uporabno obliko izhoda (imena tonov, tablatura, itd.). Podrobno bom opisal implementacijo le te in opisal ogrodja, s katerimi si bom pri tem pomagal. Opisal bom programski jezik Python ter knjižnjico OpenCV in utemeljil, zakaj sem ta orodja tudi izbral. Zaključil bom z rezultati in s tem, na kaj je bilo treba paziti pri testiranju ter komentiral dobljene rezultate in zadovoljstvo z zaključkom.

\bigskip

\noindent\textbf{Ključne besede:} \tkeywords.
% prazna stran
\clearemptydoublepage

%%%%%%%%%%%%%%%%%%%%%%%%%%%%%%%%%%%%%%%%
% abstract
\selectlanguage{english}
\addcontentsline{toc}{chapter}{Abstract}
\chapter*{Abstract}

\noindent\textbf{Title:} \ttitleEn
\bigskip

\noindent\textbf{Author:} \tauthor
\bigskip

%\noindent\textbf{Abstract:} 
\noindent I will present the problem of optical musical recognition. The paper will contain a description and explanation of OMR (optical music recognition) technology with examples. The result will be a mobile application that will allow you to capture notes and select a familiar / useful output format (tone names, tablature, etc.). I will describe in detail the implementation of this and describe the frameworks with which I will help myself. I will describe the Python programming language and the OpenCV library and explain why I chose these tools. I will conclude with the results and how to perform testing, with comment on the results obtained and satisfaction with the conclusion.
\bigskip

\noindent\textbf{Keywords:} \tkeywordsEn.
\selectlanguage{slovene}
% prazna stran
\clearemptydoublepage

%%%%%%%%%%%%%%%%%%%%%%%%%%%%%%%%%%%%%%%%
\mainmatter
\setcounter{page}{1}
\pagestyle{fancy}

\chapter{Uvod}
V \ref{ch1}.~poglavju bomo naredili pregled nad področjem ter posikali najbolj relavantne članke, knjige in ostale vire in utemeljili njihovo povezavo s tematiko diplomskega dela.
Sledilo bo \ref{ch2}.~poglavje, kjer bomo spoznali orodja, ki jih bom uporabljal za razvoj mobilne aplikacije; to sta programski jezik Python ter knjižnica OpenCV.
V \ref{ch3}.~poglavju bomo spoznali tehnologijo optične razpoznave znakov, nadaljevalo pa se bo z \ref{ch4}.~poglavjem, kjer bomo spoznali njeno pod vejo in sicer tehnologijo optične razpozunave glasbenih zapisov.
Nato bomo v \ref{ch5}.~poglavju vas bom vodil skozi razvoj mobilne aplikacije. Poglavjej e razdeljeno na dve sekciji in sicer na pripravo procesorja slike in modela za klasifikacijo ter testiranje.
V zadnjem, \ref{ch6}.~poglavju, bom prikazal ter komentiral rezulate.

\chapter{Motivacija}
\label{ch0}

Že od majhnih nog se ukvarjam z glasbo. Obiskoval sem glasbeno šolo, sodeloval pri raznih orkestrih, glasbenih skupinah, projektih itd. Sledilo je dolgo obdobje, ko se več nisem želel učiti igranja inštrumenta po notah in sem postal samouk. Sedaj mi je seveda žal, saj v branju not več nisem vešč in mi branje v realnem času lahko predstavlja problem. Aplikacija, ki bi omogočala razpoznavo not, bi mi zelo pomagala; ne samo z prišparanjem sramote, temveč tudi pri učenju.

Prav tako mi je vse bolj ušeč veda umetne inteligence ter strojnega učenja. S to tematiko se mi odpre možnost pridobitve tako teoretičnega kot tudi praktičnega znanja iz veje, ki me zanima, mi je zanimiva in bi mi v vsakdanjem življenju prišla prav.

Rezultati naloge, bi pa poleg meni lahko bili v pomoč tudi ostalim, bodisi glasbenikom, ki imajo težave z branjem not bodisi ljudem, ki jih zanima tehnologija optične razpoznave znakov.


\chapter{Pregled področja}
\label{ch1}

Tehnologija optične razpoznave znakov obstaja že dolgo časa, zato je veliko gradiva s katerim si bom lahko pomagal. Ker se je veda do sedaj že toliko razvila, bi težko pustil svoj pečat s to diplomsko nalogo, lahko pa zato dosti bolj prilagodim svojim potrebam in željam ter pripravim aplikacijo, ki mi bo prišla prav takoj, ko bo končana. Če bo kasneje ali aplikacija ali diplomska naloga pomagala pri čemer koli, še toliko bolje.

Gradiva v zvezi s to tematiko je veliko, izbral sem jih par, ki so se mi zdela najbolj zanimiva in uporabna za moje potrebe in želje:
\begin{itemize}
\item
Knjiga "Optical music sheet segmentation" \cite{omss}, v kateri je predstavljen je segmentacijski modul sistema O / sup 3 / MR (objektno orientirano optično prepoznavanje glasbe). Predlagani pristop temelji na sprejetju projekcij za ekstrakcijo osnovnih simbolov, ki predstavljajo grafični element glasbene notacije.
\item
Članek "The Challenge of Optical Music Recognition" \cite{tcomr}, opisuje izzive, ki jih predstavlja optično prepoznavanje glasbe. Najprej je opisan problem, nato pa je predstavljen splošen okvir za programsko opremo, ki poudarja ključne točke, ki jih je treba rešiti: identifikacija osebja, prepoznavanje glasbenih predmetov, klasifikacija glasbenih funkcij in glasbena semantika.
\item
Članek "New approaches to Optical Music Recognition" \cite{naomr}, opisujejo sistem se osredotoča na prepoznavo sestavljenih simbolov (akordi in skupine snopov).
\end{itemize}


\chapter{Uporabljene metode}
\label{ch2}

Pri izdelavi primerov ob razlagi tehnologij ter implemetaciji aplikacije bom uporabljal programski jezik Python ter knjižnjico OpenCV za obdelovanje slik ter uporabo metod in algoritmov za razpoznavo znakov.


\section{Python}

V tem podpoglavju bom predstavil in opisal programskegi jezik Python ter obrazložil zakaj ga bom uporabljal. Navedel bom primerjavo z ostalimi jeziki in predstavil prednosti jezika pri implementaciji aplikacije, ki uporablja tehnologijo optične razpoznave znakov.


\section{OpenCV}

V tem podpoglavju bom predstavil knjižnico OpenCV. Utemeljil bom razlog, da sem se odločil za uporabo le te, opisal njene prednosti in slabosti proti ostalim knjižnjicam. Predstavil bom funckije, ki nam jih ponuja ter zakaj in kako mi bodo te prišle prav pri implementaciji aplikacije.


\chapter{OCR}
\label{ch3}

V tem poglavju bi podrobno opisal tehnologijo razpoznave znakov. Predstavil bi začetek, kaj rešuje in kako rešuje. Omenil bi tudi praktične primere, saj bi tako bilo lažje razumeti. Proti koncu bi začel s povezavanjem OCR z OMR.


\chapter{OMR}  % poglavje dodal Solina
\label{ch4}

Opisal bi konkretnejšo problematiko in navedel primere. Ob njih bi predstavil to podvejo razpoznave znakov. Pordobneje bi opisal kako je reševanje zgrajeno


\chapter{Razvoj aplikacije} 
\label{ch5}

V tem poglavju bom bralce popeljal skozi celoten proces razvijanja mobilne aplikacije. Pisal bom o pripravljanju samega okolja, konfiguracijah ter po sklopih razdeljenemu delu.

\section{Priprava procesorja slike in modela za klasifikacijo}

V tem podpoglavju bom nazorno predstavil postopek izdelave procesnega dela aplikacije. Pokazal bom kako se pripravi slike na obdelavo in kakšen je rezultat algoritmov. Prikazal bom tudi uporabo več modelov in utemeljil izbiro najboljšega.

\section{Testiranje}

V tem podpoglavju bom opisal postopek testiranja, torej kako bo potekalo, kaj bomo potrebovali, kaj pričakujemo, kako oceniti rezultat, itd.


\chapter{Sklepne ugotovitve}
\label{ch6}

Zadnje poglavje bom posvetil končnim rezultatom in jih komentiral. Utemeljil bom, kako uspešni so bili rezultati, ter komu bi lahko prišli prav. Napisal bom tudi kako sem sam zadovoljen z rezultatom ter delom nasplošno.

Zapisal bom tudi kaj sem od celotnega procesa (izbira teme, mentorja, pisanje diplome, testiranje) naučil in kako bom lahko znanje uporabil v bodočnosti.


\newpage %dodaj po potrebi, da bo številka strani za Literaturo v Kazalu pravilna!
\ \\
\clearpage
\addcontentsline{toc}{chapter}{Literatura}
\bibliographystyle{plain}
\bibliography{literatura}


\end{document}

